\documentclass{article}
\usepackage{amsmath, amssymb, amsfonts, amsthm}
\usepackage[margin=1in]{geometry}
\usepackage{fancyhdr}
\usepackage[final]{pdfpages}
\usepackage[makeroom]{cancel}
\usepackage{enumitem}

\usepackage[normalem]{ulem} 

\newcommand{\lecture}[3]{
   \pagestyle{myheadings}
   \thispagestyle{plain}
   \newpage
   \setcounter{page}{1}
   \noindent
   \begin{center}
   \framebox{
      \vbox{\vspace{2mm}
      \hbox to 6.28in { {\bf CSCI-4968: ML and Optimization}
      \hfill Spring 2023 }
        \vspace{5mm}
        \hbox to 6.28in { {\Large \hfill #1 \hfill} }
        \vspace{2mm}
         \hbox to 6.28in { {\it #2 \hfill #3 \ \ \ } }
        % \vspace{2mm}
      }
   }
   \end{center}
   \markboth{Lecture #1: #2}{Lecture #1: #2}
}

\begin{document}

\lecture{Semidefinite Programming}{}{For approximating Max Cut and Travelling Salesman}

\section{Overview}

NP-Complete Problem refers to a set of very "hard" problems that are verifiable in polynomial time. 
It is largely believed there exists no polynomial time solution, 
\subsection{Max Cut} % algorithm listings
We would like to partition our graph into two set of vertices such that the sum of the edge weights between the two partitiions is maximized. 

Applications include 
examples: Clustering
- Intuition: The distances between the two partitiions are the greatest so they have a high likelihood of belonging to two different categories. 

\subsection{Travelling Salesman Problem} % algorithm listings
Imagine you are prospective student touring RPI. You have a list of buildings you want to visit.
You have on hand the distance between each pair of buildings. 
Is it possible to find a path such that each buidling is visited exactly once? 
If so, what is the shortest possible distance you will need to travel. 
Since you toruing the college, it is likely you have some method of transportation to came from and need to come back to . 
We have the added constraint that you start at the parking lot and will need to come back to the parking lot.
This is know as the travelling salesman.
% One sentence Goal
As our goal is to find an \emph{efficient} route, we will be satisfied with an approximation. 
Or a series of buidlings to visit that is close the optimal.


% How is this different form the max cut 

\section{Semidefinite Programming}  % objective functions

\subsection{Recap of Linear Programming}
To give background on semi-definite programming we first start with a brief recap of linear programming. Suppose that you have control over a set of variables and you are attempting to find a selection for each of these variables such that some linear combination is maximized / minimized. To make this more complicated suppose that these variables each have a constraint that must be met.

Formally we can write this out as the following. Suppose that we have \( n \) variables that we have control over which we can represent as:
\[
\vec{x} = \begin{bmatrix} x_1 \\ x_2 \\ \vdots \\ x_n \end{bmatrix}
\]

Say for each of these \( n \) variables we have an associated coefficient \( \vec{c} = \begin{bmatrix} c_1, c_2, ..., c_n  \end{bmatrix}^\top \) and in the end we want to pick \( \vec{x} \) such that the linear combinatrion of the two (\( \vec{c} \cdot \vec{x} \)) is
\[
\min \vec{c} \cdot \vec{x}
\]

Without constraints the problem is trivial as we just set \( \vec{x} \) to an arbitrarily large / small value depending on the problem. Suppose that we must also satisfy a system of equations. We let \( \vec{a}_i \) and \( \vec{b} \) be a \( n \) length vectors. Then we want our pick of \( \vec{x} \) to satisfy:
\begin{align*}
\vec{a}_1 \cdot \vec{x} &= b_1 \\
\vec{a}_2 \cdot \vec{x} &= b_2 \\
& ...  \\
\vec{a}_n \cdot \vec{x} &= b_n \\
\end{align*}

We can gather our \( \vec{a}_i \) vectors into a single matrix so that we only need to solve:
\[
\mathbf{A} \vec{x} = \vec{b}
\]

In total we can define a linear programming problem as the following where we want to find an optimal \( \vec{x} \):
\begin{align*}
  \min \vec{c} \cdot \vec{x} \\
  \mathbf{A} \vec{x} = \vec{b} \\
\end{align*}


What is nice about linear programming is that it is a convex optimization problem. Finding a solution to an instance of linear programming can be done with a variety of algorithms including *simplex algorithm* \textbf{\textcolor{red}{TODO: Double check this}}.

This is what linear programming solves and is usable in a large variety of problems such as
\textbf{\textcolor{red}{TODO: Need to find more examples}}.

\subsection{Formalization of Semidefinite Programming}
Instead of having just \( n \) variables to pick suppose that we have \( n^2 \) variables. Thus instead of finding a \( \vec{x} \) we are trying to find \( \mathbf{X} \). With more variables comes more constraint so suppose that we \( n \) equations to satisfy. For each equation we have a matrix \( \mathbf{A}_i \) and a scalar value \( b_i \) defining our constraint. Then for a given constraint we have:
\[
\mathbf{A}_i \bullet \mathbf{X} = b_i
\]

Where the \( \bullet \) operator is simply taking the sum of all elements after doing element wise multiplication on matricies two matricies \( \mathbf{A} \) and \( \mathbf{X} \) with the same shape. Formally we can write it as:
\[
  \mathbf{A} \bullet \mathbf{X} = \sum_{i = 1}^n \sum_{j = 1}^n \mathbf{A}_{i, j} \mathbf{X}_{i, j}
\]
An easy way of

In total we  can write out a rigorous form of Semdefinite programming.
\begin{align*}
  \min \mathbf{C} \bullet \mathbf{X} \\
  \mathbf{A}_i \bullet \mathbf{X} &= b_i  \text{   for } i = 1, 2, ..., n\\
  \mathbf{X} \succeq 0 \\
\end{align*}


\section{Relaxation}   % objective functions
\subsection{Proofs}
- performance guarantees
- proof of relaxations (how accurate + how it works)


Integrality Gap: ratio btwen optimal / relaxation 



\section{Experimental Results}
\subsection{Visualization}
- table of result
- for small examples, we find the acutal solution (brute force), or a solution (integer programming) and check how 
good our bound is

vhttps://www.cs.cmu.edu/~anupamg/adv-approx/lecture14.pdf used "randomized rounding" to find a max cut.
i.e. if $p_{ij}$ is close to -1, then we should cut it. We attempt do do something similar

\section{Discussion}


- can we prove the runtime
- can we prove how good of a lower bound we have?
- what are some easy things to prove?

\subsection{Scope and limitations}

\section{References}
\end{document}
